\section{Related Work}

Many works have attacked this problem using different design techniques.
In Kandoo{[}{]}, the authors propose a hierarchical controller design,
where the local applications are offloaded to the local controllers,
while the applications requiring global view of the network, execute
on a centralised one. The design requires maintenance of complex data
structures between the global and local controllers. The Difane{[}{]},
presents a TCAM expensive solution, where the secondary switches acts
as cache devices. It partitions the flows among a set of switches
and installs appropriate rules, to selectively direct packets to specific
switches. NOSIX aids the applications running on the controller with
a view of the virtual flow tables. However, it relies on the vendor
to provide an interface of updating virtual to physical tables. The
work on {[}{]}(Rexford Paper) tries to solve the \textquotedblleft{}infinite
flow\textquotedblright{} capacity problem, by extending its work on
the Difane paper. It exploits the large data capacity of software
switches by using them as the secondary cache devices. It requires
the complex overhead of creating it\textquoteright{}s own rules which
forms the basis of selectively directing packets to the specific software
switches on a hardware switch miss.

In our SDN architecture, the flowcache acts a proxy device which avoids 
the complexity of creating new rules by installing rules which are inserted 
by the controller. In this paper, we present a novel architecture, in line 
with the memory system architecture to achieve the required performance goals.

