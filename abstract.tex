%%%%%%%%%%%%%%%%%%%%%%%%%%%%%%%%%%%%%%%%%%%%%%%%%%%
%
%  New template code for TAMU Theses and Dissertations starting Fall 2012.  
%  For more info about this template or the 
%  TAMU LaTeX User's Group, see http://www.howdy.me/.
%
%  Author: Wendy Lynn Turner 
%	 Version 1.0 
%  Last updated 8/5/2012
%
%%%%%%%%%%%%%%%%%%%%%%%%%%%%%%%%%%%%%%%%%%%%%%%%%%%
%%%%%%%%%%%%%%%%%%%%%%%%%%%%%%%%%%%%%%%%%%%%%%%%%%%%%%%%%%%%%%%%%%%%%
%%                           ABSTRACT 
%%%%%%%%%%%%%%%%%%%%%%%%%%%%%%%%%%%%%%%%%%%%%%%%%%%%%%%%%%%%%%%%%%%%%

\chapter*{ABSTRACT}
\addcontentsline{toc}{chapter}{ABSTRACT} % Needs to be set to part, so the TOC doesnt add 'CHAPTER ' prefix in the TOC.

\pagestyle{plain} % No headers, just page numbers
\pagenumbering{roman} % Roman numerals
\setcounter{page}{2}

\indent 
Virtualization  technology is powering today's cloud industry. Virtualization inserts a software layer, the hypervisor, below the  Operating System, to manage multiple OS environments simultaneously. Offering numerous benefits such as fault isolation, load balancing, faster server provisioning, etc., virtualization occupies a dominant position, especially in IT infrastructure in datacenters. Memory management is one of the core components of a hypervisor. Current implementations assume the underlying memory technology to be homogenous and volatile. However, with the emergence of NVRAM in the form of Storage Class Memory, this assumption remains no longer valid. New motherboard architectures will support several different memory classes each with distinct properties and characteristics. The hypervisor has to recognize, manage, and expose them separately to the different virtual machines. This study focusses on building a separate memory management module for Non-Volatile RAM in Xen hypervisor. We  show that it can be efficiently implemented with few code changes and minimal runtime performance overhead.
